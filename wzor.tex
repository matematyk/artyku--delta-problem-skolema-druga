\documentclass[leqno,10pt]{article}
\usepackage{algorithm}
\usepackage[noend]{algpseudocode}
\usepackage{hyperref}
\def\Z{\mathbb Z}
\def\Q{\mathbb{Q}}
\def\A{\mathbb{A}}

\makeatletter
\renewcommand{\ALG@name}{Algorytm}
\makeatother

\usepackage{amsmath}
\usepackage{float}
\usepackage{delta}
\def\marg#1{\marginpar{\scriptsize\raggedright#1}}

\begin{document}

\wtyt{Problem Skolema}

\waut{Marcin Wierzbiński*}

\marg{Student, Wydział Matematyki, Informatyki i Mechaniki, Uniwersytet Warszawski}

Wiele matematycznych problemów ma bardzo proste sformułowanie, ale ich istota dotyka głębokiej matematyki. Przykładem takiego problemu jest tytułowy problem Skolema. Sformułowanie wydaje się bliskie informatyce teoretycznej, zainteresowany Czytelnik znający tzw. problem stopu, może zauważyć tutaj analogię. Natomiast techniki służące do rozwiązania problemu pochodzą z algebry i, na dodatek, pełne rozwiązanie wymagałoby istotnych postępów w tej dziedzinie.

Zacznijmy od rozważenia takiego oto pytania: Czy dany liniowy ciąg rekurencyjny (np. ciąg Fibonacciego (\ref{fib})) ma pewien wyraz równy zero? \marg{
Kolejne wyrazy ciągu Fibonacciego: $0, 1, 1, 2, 3, 5, 8, 13, \ldots$
} Każdy bez trudu odpowie na to pytanie dla ciągu Fibonacciego. Tylko dla $n=0$ wyraz $F_0 = 0$ jest równy zero. 
\begin{equation}\label{fib}
    \begin{cases}
        F_{n} = F_{n-1} + F_{n-2} \\
        F_{1} = 1 \\ 
        F_{0} = 0
    \end{cases}
\end{equation}
Kolejne wyrazy ciągu Fibonacciego $F_n$ zawsze będą dodatnie (suma wyrazów dodatnich jest dodatnia). Nie potrzebujemy do tego zaawansowanej matematyki. 
Czy odpowiedź na to pytanie dla dowolnego ciągu jest taka prosta?
Rozważmy na przykład następujący ciąg: \marg{Kolejne wyrazy dla tego ciągu to: 0, 1, 2, 1, -4, -11, -10, 13, 56, 73, -22, -263 \ldots}
\begin{equation}\label{ref:fib}
    \begin{cases}
         u_n = 2 u_{n-1} - 3 u_{n-2} \\
         u_1 = 1 \\
         u_0 = 0
    \end{cases}
\end{equation}

Na pierwszy rzut oka problem nie jest łatwy do rozstrzygnięcia. Wypisując kolejne wyrazy, możemy wysnuć hipotezę, że owy ciąg również poza pierwszym nie ma wyrazu zerowego. Problem można sformułować w ogólności, dla większej ilości równań. W tym artykule postaramy się pokazać, że już dla prostych równań za rozwiązaniem stoi ciekawa matematyka. 
    

\vfill
\mtyt{Problem Skolema}

\marg{Liniowa rekurencja jest fundamentem wielu pojęć w informatyce czy kombinatoryce. 
}

Wprowadźmy potrzebne definicje. Liniowy ciąg rekurencyjny to ciąg liczb całkowitych $u_0, \ldots, u_{k-1} \in \mathbb{Z}$, taki, że dla pewnych $a_0 \neq$ 0 $a_0, \ldots, a_{k-1} \in \mathbb{Z}$ oraz dla każdego $n \in \mathbb{N}$, $n \geq k$ mamy $u_{n}=a_{0} u_{n-1}+a_{1} u_{n-2}+\ldots+a_{k-1} u_{n-k}$. Rząd takiego ciągu to $k$.

\textbf{Problem Skolema}, to pytanie, czy dla danego liniowego ciągu rekurencyjego, istnieje takie $n$, że $u_n = 0$?
\marg{
Albert Thoralf Skolem (ur. 23 maja 1887 w Sandsvaer, zm. 23 marca 1963)
}

Obecnie wiadomo, że problem Skolema jest rozstrzygalny dla ciągów rzędu $2$, $3$ i $4$, natomiast rozstrzygalność dla rzędu $5$ pozostaje nadal otwarte. Znamy jednak rozstrzygalność różnych podprzypadków nawet dla rzędu $5$. 

Okazuje się, że już dla równań rzędu $2$ rozwiązanie jest nietrywialne. W artykule postaramy się przedstawić intuicje stojące za ową nietrywialnością. Bardzo przydatna będzie znany zapewne niektórym Czytelnikom własność ciągów rekurencyjnych.

\textbf{Twierdzenie:}
Dla każdego liniowego ciągu rekurencyjnego $u_n$ rzędu $k$ można obliczyć
liczby zespolone $\lambda_1, \ldots, \lambda_j$
oraz wielomiany $p_1, \ldots, p_j$
takie, że
\[
u_n = p_1(n) \lambda_1^n + p_2(n) \lambda_2^n + … + p_j(n) \lambda_j^n,
\]
oraz
\[
\deg(p_1) + \ldots + \deg(p_j) \leq k,
\]
gdzie przez $\deg(p)$ oznaczamy stopień wielomianu $p$.


Dla zainteresowanego Czytelnika znającego podstawowe narzędzia z algebry liniowej to twierdzenie nie jest trudno udowodnić. 


\vfill
\mtyt{Równania rekurencyjne rzędu 2}


W dalszej części przedstawię techniki związane z rozwiązaniem problemu Skolema dla liniowego ciągu rekurencyjnego rzędu 2. Rozwiązanie będzie zależało od współczynników, które występują w twierdzeniu powyżej. Prześledzimy je na kilku przykładach, które zilustrują techniki rozwiązywania naszego problemu w różnych przypadkach.

\textbf{Różne moduły $\lambda_1$ i $\lambda_2$}\\
Rozważmy jeszcze raz ciąg Fibonacciego. Na początek zobaczmy jak uzyskać wzór jawny na $n$-ty wyraz ciągu. Stosując twierdzenie wiemy, że $n$-ty wyraz ciągu Fibonacciego:
\begin{equation*}
    F_{n} = \frac{\lambda_1 \lambda_2^{n+1}-\lambda_2 \lambda_1^{n+1}}{\sqrt{5}} = \frac{\lambda_1^{n+1}\lambda_2\left(\left(\frac{\lambda_2}{\lambda1}\right)^{n}-1\right)}{\sqrt{5}}
\end{equation*}
Warto zaznaczyć, że jawne obliczenie $\lambda_1$ i $\lambda_2$ to prześledzenie rozumowania dowodu twierdzenia i wykorzystania reprezentacji liniowego ciągu rekurencyjnego oraz sprawdzenie wartości własnych pewnej macierzy.  
 

Możemy teraz ponownie (lecz inaczej) wykazać, że ciąg Fibonacciego nie ma wyrazów zerowych poza $F_0$. Wystarczy zaobserwować, że $F_n = 0$ wtedy i tylko wtedy, gdy $(\frac{\lambda_2}{\lambda_1})^{n}=1$. Można łatwo zauważyć, że nie jest to możliwe dla $n > 0$, ponieważ $|\lambda_1| > |\lambda_2|$.
Podobnymi technikami można pokazać, że dla dowolnego ciągu, którego macierz ma dwie wartości własne o różnych modułach od pewnego momentu wyrazy nie mogę być równe zero. Wówczas do sprawdzenia, czy istnieje jakikolwiek wyraz zerowy wystarczy jedynie zbadać zerowość pewnej liczby początkowych wyrazów ciągu.

\textbf{Równe współczynniki $\lambda_1$, $\lambda_2$}\\
Rozważmy kolejny przykład ciągu:
\begin{equation}\label{przyk:3}
    \begin{cases}
        u_n = -2 u_{n-1} - u_{n-2} \\
        u_1 = 0 \\
        u_0 = 1
    \end{cases}
\end{equation}
Kolejne wyrazy tego ciągu to $1, 0, -1, 2, -3, 4, -5, 6, -7, 8 \ldots$, więc również spodziewamy się, że tylko $u_1 = 0$.

Podobnie jak poprzednio, opierając się o twierdzenie możemy jawnie wyrazić wyraz $u_{n}$ jako:
\begin{equation*}
    u_n = \lambda_{1}^n + \lambda_{1}^{n-1} n
\end{equation*}
Wykazanie, czy istnieje $n$ takie, że $u_n=0$ staje się proste. A zatem $u_n = 0$ jeśli $\lambda_1 + n = 0$, czyli gdy $n = -\lambda_1 = 1$.

Nietrudno pokazać, że z kolei tą metodą można rozwiązać dowolny przypadek, gdy mamy dwukrotną wartość własną.

\textbf{Ostatni przypadek}\\
Przed nami najciekawsza część, przypadek gdy $|\lambda_1| = |\lambda_2|$, ale $\lambda_1 \neq \lambda_2$.
Rozważmy nasz przykład (\ref{ref:fib}) ze wstępu. 
Obliczenie współczynników daje: $\lambda_1 = 1 - i\sqrt{2}$, $\lambda_2 = 1 + i \sqrt{2}$. Zauważmy, że co do modułów wartości własne są równe. 
Okazuje się, że ten przypadek jest szczególnie interesujący i wymaga trochę więcej uwagi. 

Nasz liniowy ciąg rekurencyjny można zapisać w postaci\marg{Zainteresowany Czytelnik może przeczytać o liczbach zespolonych: \href{http://www.deltami.edu.pl/temat/matematyka/algebra/2016/09/30/Liczby_zespolone_i_kwaterniony/}{Delta, październik 2016:
Liczby zespolone i kwaterniony}}
\begin{equation*}
    u_{n} = \frac{i}{2 \sqrt{2}}(\lambda_{1}^{n}- \lambda_2^{n})
\end{equation*}
Przeniesiemy teraz nasze rozważania na grunt uogólniony. 

Mianowicie można nietrudno pokazać, że nasze liniowy ciąg rekurencyjny będzie przyjmował postać $u_n = a \lambda_1^{n} + \overline{a} \lambda_2^{n}$, dla pewnej liczby algebraicznej $a \in \mathbb{A}$ i \marg{
    Powiemy, że $a $ jest liczbą algebraiczną, jeżeli jest pierwiastkiem niezerowego wielomianu o współczynnikach wymiernych. Zbiór liczb algebraicznych oznaczmy jako $\mathbb{A}$. Liczba $\sqrt{2}$ jest liczbą algebraiczną wielomianu $p(x) = x^{2} - 2$
} liczb zespolonych takich, że $\lambda_1 = \overline{\lambda_2}$.

Pytając się o warunek $u_n = 0$, możemy zauważyć, że $a \lambda_1^{n} + \overline{a} \overline{\lambda_1}^{n} = 0 $ wtedy i tylko wtedy, gdy cześć rzeczywista $a {\lambda_1^{n}}$ jest równa $0$. 
Weźmy: $v = \frac{\lambda_1}{|\lambda_1|} $ wówczas $|v| = 1$.
Wystarczy sprawdzić, czy $\frac{u_n}{|\lambda_1|^n} = 0$, czyli, czy $a v^{n} + \overline{a} \overline{v} ^{n} = 0$.
To zaś jest równoważne temu, że $a v^{n}$ jest czysto urojone (postaci $ix$ dla $x \in \mathbb{R}$). Ponieważ $|v| = 1$, to musi być $x = |a|$. Pytamy więc, czy istnieje takie $n$, że $v^{n} = \frac{i |a|}{a}$.

Pozostaje nam rozwiązać równanie postaci
\begin{equation*}
    v^{n} = \beta, \text{ gdzie } v, \beta \in \A, |v| = |\beta| = 1.
\end{equation*}
To w ogólności wymaga pochylenia się nad teorią liczb algebraicznych. Jednak dla sprawnego algebraika, takie równania nie stanowią problemu. Dla ciągów rekurencyjnych rzędu $3$ również znamy algorytm bazujący na podobnym rozumowaniach, jednak w ogólności dla $k>5$ nie jest znane żadne rozwiązanie problemu Skolema. 



\end{document}

